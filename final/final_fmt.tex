\documentclass[12pt]{article}


\usepackage{fontspec}
\usepackage{polyglossia}

\setmainlanguage{russian}

\setmainfont{Linux Libertine O}
\setsansfont{Linux Libertine O}


\usepackage{amsmath}
\usepackage{amssymb}
\usepackage{geometry}
\usepackage{graphicx}
\usepackage{sidecap}
\usepackage{verbatim}

\geometry{top=0.5cm, bottom=0.5cm, left=1.5cm, right=2cm} %поле сверху

\usepackage{wrapfig}
\usepackage{epigraph}

\pagestyle{empty}
\DeclareGraphicsExtensions{.pdf,.png,.jpg}

\begin{document}



\begin{center}
\begin{tabular}{cc}
\includegraphics[scale=0.85]{klsh_logo_mod.pdf} &
\raisebox{0.6cm}{
  {\Large\bf Суперфинал и финал}
}
\end{tabular}
\end{center}
\vspace{-1cm}
{\noindent\it Желаем успехов!} 


\begin{enumerate}
\item[\bf 1.] Сторона куба равна 5. В центре каждой грани куба вырезают квадратную дырку размером $2 \times 2$. 
Дырки сквозные, их стороны параллельны соответствующим рёбрам куба. Найди объем оставшейся части куба.

\item[\bf 2.] Рыбак находится на льдине, верхняя поверхность льдины находится над водой. 
Льдина имеет вид вертикального цилиндра. Определи наименьшую возможную площадь льдины, если масса рыбака — $m$, 
а толщина льдины — $h$. Плотность воды равна $\rho_1$, плотность льда — $\rho_2$, где $\rho_1 > \rho_2$. 
Ускорение свободного падения равно $g$.

\item[\bf 3.] В треугольнике $\bigtriangleup ABC$ сторона $BC$ равна $2 \sqrt{3} / 3$. 
Медианы треугольника $A A_1$,\,$B B_1$,\,$C C_1$ пересекаются в точке $O$, и известно, 
что точки $O$,\,$B_1$,\,$C_1$,\,$A$ лежат на одной окружности. Найди длину медианы $A A_1$.

\item[\bf 4.] Подвешенному на нити шарику сообщили начальную скорость в горизонтальном направлении. 
Когда нить отклонилась на угол $\alpha = \pi/6$ от вертикали, ускорение шарика оказалось направленным горизонтально. 
Найди $\cos\beta$, где $\beta$ — это угол максимального отклонения~нити.

\end{enumerate}

\begin{center}
\begin{tabular}{cc}
\includegraphics[scale=0.85]{klsh_logo_mod.pdf} &
\raisebox{0.6cm}{
  {\Large\bf Суперфинал и финал}
}
\end{tabular}
\end{center}
\vspace{-1cm}
{\noindent\it Желаем успехов!} 


\begin{enumerate}
\item[\bf 1.] Сторона куба равна 5. В центре каждой грани куба вырезают квадратную дырку размером $2 \times 2$. 
 Дырки сквозные, их стороны параллельны соответствующим рёбрам куба. Найди объем оставшейся части куба.
  
\item[\bf 2.] Рыбак находится на льдине, верхняя поверхность льдины находится над водой. 
Льдина имеет вид вертикального цилиндра. Определи наименьшую возможную площадь льдины, если масса рыбака — $m$,  
а толщина льдины — $h$. Плотность воды равна $\rho_1$, плотность льда — $\rho_2$, где $\rho_1 > \rho_2$. 
Ускорение свободного падения равно $g$.
 
\item[\bf 3.] В треугольнике $\bigtriangleup ABC$ сторона $BC$ равна $2 \sqrt{3} / 3.$ 
Медианы треугольника $A A_1$,\,$B B_1$,\,$C C_1$ пересекаются в точке $O$, и известно, что точки $O$,\,$B_1$,\,$C_1$,\,$A$ лежат на одной окружности. 
Найди длину медианы $A A_1$.

\item[\bf 4.] Подвешенному на нити шарику сообщили начальную скорость в горизонтальном направлении. 
Когда нить отклонилась на угол $\alpha = \pi/6$ от вертикали, ускорение шарика оказалось направленным горизонтально. 
Найди $\cos\beta$, где $\beta$ — это угол максимального отклонения~нити.

\end{enumerate}

\newpage


Турнир в 15:10, Прага, суперфинал: $\chi$ — $\sigma$. Тарту, финал: $\pi$ — $\xi$.

\begin{enumerate}
  \item вырезаемая часть куба равна $60 - 16 = 44$, оставшая часть куба равна  $125 - 44 = 81$
  \item 
  \[ 
    mg + \rho_2 h Sg = \rho_1 h S g
  \]
  \[
    S = \frac{m}{h(\rho_1 - \rho_2)}
  \]
  \item Углы $\angle OAB_1$ и $\angle OC_1B$ равны. Опираются на одну дугу. 

  Треугольники $\triangle OA_1C$ и $\triangle CA_1A$ подобны.
  
  Пусть $OA_1 = x$, $BC = a$, тогда 
  
  \[
  \frac{x}{a/2} = \frac{a/2}{3x}  
  \]
  
  $a = 2\sqrt{3}x$
  
  $x=1/3$
  
  Ответ: 1.
  
  \item Второй закон Ньютона:
  \[
  \frac{mv^2}{R} = T - mg \frac{\sqrt{3}}{2} 
  \]
  Горизонтальное ускорение:
  \[
  mg = T \frac{\sqrt{3}}{2}  
  \]
  Получаем 
  \[
  \frac{v^2}{R} = \frac{2g}{\sqrt{3}} - g \frac{\sqrt{3}}{2}  
  \]

  Закон сохранения энергии
  \[
  \frac{v^2}{2} +gR\left(1- \cos \frac{\pi}{6} \right)  = gR(1-\cos\beta)  
  \]
  Делим на $R/2$ и подставляем:
  \[
  \frac{v^2}{R} +2g(1- \sqrt{3}/2)  = 2g(1-\cos\beta)  
  \]

  Итого

  \[
  \frac{2}{\sqrt{3}} - \frac{\sqrt{3}}{2} - \sqrt{3} = -2\cos\beta  
  \]

  \[
  \cos\beta = \frac{5\sqrt{3}}{12}
  \]

\end{enumerate} 

\end{document}
